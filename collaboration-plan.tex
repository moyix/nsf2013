\documentclass[letterpaper,11pt]{scrartcl}

\PassOptionsToPackage{hang,small}{caption}

\usepackage[margin=1in]{geometry}
\usepackage{amsmath,amsfonts,amssymb,amsthm}
\usepackage{graphicx}
\usepackage[english]{babel}
\usepackage{subfig}
\usepackage{booktabs}
\usepackage{fancyvrb}
\usepackage{xspace}
\usepackage{scrpage2}
\usepackage{pgf,tikz}
\usetikzlibrary{calc}
\usetikzlibrary{arrows}
\usepackage{ulem}
\normalem
\usepackage{fontspec}
\usepackage{xunicode}
\usepackage{xltxtra}
\defaultfontfeatures{Mapping=tex-text}
\setromanfont[Ligatures={TeX}]{Times New Roman}
\setsansfont[Ligatures={TeX}]{Helvetica Neue}
\setmonofont[Scale=MatchLowercase]{Menlo}

\usepackage[unicode,bookmarks,colorlinks,breaklinks,pdftitle={},pdfauthor={}]{hyperref}
\hypersetup{linkcolor=black,citecolor=black,filecolor=black,urlcolor=black}

\pagestyle{scrheadings}
\clearscrheadfoot
\lohead{NSF 13-578 --- Secure and Trustworthy Cyberspace}
\rohead{\pagemark}
\lehead{\thetitle}
\rehead{\pagemark}
\setheadsepline{0.5pt}
\setfootsepline{0pt}
\setkomafont{pageheadfoot}{\sffamily\fontsize{9}{9}\selectfont}
\setkomafont{pagenumber}{\sffamily\fontsize{9}{9}\selectfont}

\newcommand{\thetitle}{TWC: Medium: Automated Reverse Engineering of Commodity Software\xspace}

\begin{document}

\title{Collaboration Plan}
\subtitle{\thetitle}
\author{}
\date{}
\maketitle

\section{Introduction}
\label{sec:introduction}

In this document, we outline the plan for facilitating collaboration between
the investigators' respective institutions.  We discuss the distribution of
project roles, project management structure, concrete mechanisms to coordinate
collaboration between each group, and budget line items supporting the plan.

\section{Project Roles and Management}
\label{sec:roles}

PIs Kirda and Robertson at Northeastern University will lead the project. They
will oversee the development of techniques for structural decomposition,
behavioral inference, and runtime instrumentation of closed-source software
systems.  They will also contribute to the development of the dynamic analysis
framework and behavioral program summaries.  Additionally, PI Kirda will
assume reporting and compliance duties for the project as a whole.

PI Wang will lead the Georgia Tech effort on the project.  He will oversee the
development of the whole-system dynamic analysis framework, behavioral program
summaries, and protocol fuzzing components.

\section{Coordination Mechanisms}
\label{sec:coordination}

As the various components of the project are closely intertwined -- e.g., all
components rely on the development of the whole-system dynamic analysis
capability -- close coordination will be required to ensure the successful
completion of the project goals.  To that end, the investigators and their
respective groups will make use of a number of coordination mechanisms.  This
includes the use of shared source code repositories, documentation, and
testing infrastructure.  In addition, the investigators will hold regular
project meetings  in order to synchronize efforts on both sides.  The
investigators will also explore potential opportunities for graduate student
visits between the two groups.

\section{Budget}
\label{sec:budget}

To support collaboration between the two groups, the budget includes line
items for travel that will allow for in-person meetings between the
investigators and research personnel involved in the project.  All other
collaboration will be supported by existing infrastructure.

\end{document}
