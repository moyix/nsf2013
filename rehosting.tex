The devices that surround us every day, from cars to coffee makers,
contain complex software that has received little public examination.
Because these devices are increasingly being connected to the Internet,
it is vital that their security be rigorously evaluated.

To date, the analysis of embedded firmware has required source
code~\cite{davidson:2013:fie}, used only static
analysis~\cite{dreissen:2012:satphone}, or applied only to Linux-based
firmware. Many advanced analyses (e.g., concolic
execution~\cite{godefroid:2005:dart}, dynamic taint analysis, and
fuzzing) require dynamic analysis. Embedded devices run a wide variety
of operating systems (Windows CE, QNX, VxWorks, and Cisco IOS, to name
but a few), and so to have broad impact we must find analyses that do
not depend on a single operating system.

To effectively support sophisticated dynamic analyses for real embedded
systems, we need a platform that can run embedded firmware in a
whole-system emulator such as \dynamicsys, allowing full visibility
into all code running on the system. Although emulators such as QEMU
support a large number of processors such as ARM, MIPS, and PowerPC, a
given embedded device firmware will typically not run due to missing
support for peripheral devices.

Missing peripheral support poses an especially severe problem from a
security standpoint. Because peripherals are the way embedded systems
interact with the outside world, an analysis tool that cannot inspect
the device drivers for the platform will miss most of the attack
surface. In order to close this gap, we propose to develop automated
techniques for building models of embedded peripherals that can then be
used by \dynamicsys to run embedded firmware.

\subsection{Assumptions and Prerequisites}

Our proposed work makes the following assumptions:%
\begin{inparaenum}[\itshape a\upshape)]
    \item it is possible to obtain a copy of the target firmware, e.g.
    by extracting it from flash memory or via the manufacturer's update
    tool;
    \item the CPU architecture on which the firmware runs can be
    determined, and the target emulator supports this CPU architecture;
    and,
    \item the load address and entry point of the firmware in memory can
    be found.
\end{inparaenum}

We believe that these assumptions are reasonable for many consumer
devices, and in any case solving these problems is orthogonal to the
main task of firmware emulation.

\subsection{Technical Approach: Software-Guided Emulation}

Our key insight is that \emph{firmware itself contains a rough
specification of the hardware on which it runs}: the checks and
operations performed on hardware-supplied values can be interpreted as
constraints on those same values. Thus when a poorly emulated piece of
hardware misbehaves, the way the firmware acts in response can to
some extent be used as an oracle for the correct behavior.

To make this idea more concrete, consider an example where firmware
reads a value from memory-mapped device memory and checks whether its
value is less than \texttt{0x100}. If value returned by emulated
hardware is greater than this value, the firmware may print an error
message and halt. Upon detecting this error, our system could then infer
the constraint and retry the emulation with a response that is in range
and allows execution to continue.

However, as this example makes clear, we need a way to detect when an
emulated execution has gone astray. Also, iteratively finding
conditionals in the firmware that relate to hardware constraints may be
intractable computationally, as the search space is potentially
exponential.

To solve these problems, we propose to develop ways of inferring the
actual device's execution path through the firmware image. For example,
if the device has a serial port that produces debug messages, these
strings can be matched within the firmware binary find the code that
emits them. Thus, one can use a sequence of debugging messages as a
``trail of breadcrumbs'' to infer a valid path through the firmware
image. This means that the symbolic execution can likewise follow this
path through the firmware, collecting constraints on the device inputs
along the way.

Additional research challenges remain, however. Although the process
described above will work for simple peripherals, devices that keep
complex state or that do computation of their own outside the CPU will
be difficult to handle. In addition, the use of interrupts and direct
memory access (DMA) by peripherals is not handled by this approach, and
new advances are needed. We hope that by overcoming these research
challenges we can arrive at a system which automatically generates
reasonable device models for many peripherals, reducing the amount of
manual reverse engineering required to a minimum.

\subsection{Preliminary Results: Modeling Simple Devices}

We have performed some initial investigation into this problem using the
firmware of an ARM-based HP LaserJet 4050dn printer. On the device, we
were able to locate a serial port, which provided a stream of boot
messages to act as the ``breadcrumbs'' described in the previous
section. We were also able to extract the firmware image from the
devices and load it in \dynamicsys.

By combining \dynamicsys with a symbolic execution engine and SAT
solver, we were able to \emph{automatically} infer peripheral device
models for many of the devices used in the early stages of boot,
including the serial port and NVRAM configuration. Our automatically
generated device models allow the system to partially boot, reproducing
the real serial console output, until it reaches code that stops the
system, waiting for an interrupt.

For comparison, before trying to automatically generate peripheral
device models we also attempted to create them by hand, a process which
took several weeks to reach a similar state --- a sharp contrast with
the few minutes of computation required for our automated system.

These results indicate that the approach described in the previous
section has the potential for reducing the work required to emulate an
embedded firmware, but many interesting research problems are still
unsolved.

%For example, one could use JTAG to create a device snapshot that
%could be revived in QEMU, allowing the analysis described above to skip
%over potentially complex device initialization code and more quickly get
%to the business of testing the attack surface of the device.
%Alternatively, JTAG could be used to sample the device's execution,
%allowing direct comparison with the emulated version; this ``trail of
%breadcrumbs'' could even be used as input to a constraint solver that to
%identify what device inputs are needed to reach the sampled code
%paths.\footnote{This last approach draws on techniques recently proposed
%by Jin and Orso~\cite{Jin2012} to reproduce software failures seen in
%the field.}

%. Many embedded devices
%support IEEE 1149.1, the Standard Test Access Port and Boundary-Scan
%Architecture~\cite{jtag}, more commonly known as JTAG. The JTAG
%interface of many embedded devices offers some amount of debugging
%support, and allows one to halt the processor, read memory and CPU
%state, and so on. Such information could be used by an automated
%rehosting solution in a number of ways, as we detail below.
%
%First and most trivially, the ability to peek into the runtime state of
%a correctly functioning device would allow features such as the location
%and contents of the firmware to be determined directly, which would make
%the initial phase of the procedure described above more efficient.
%
%The JTAG interface could also be used to sample the execution of the
%real device during boot. This would provide a ``trail of breadcrumbs''
%that the emulator could then follow. When the boot fails in the
%emulator, these sample points could guide the search for a solution.
%For example, if the emulated boot hangs in a loop while waiting for an
%interrupt from a non-existent device, the execution samples might
%indicate that the next code to be executed lies in an interrupt handler,
%which would allow the rehosting assistant to create a stub that asserts
%the appropriate interrupt. Similarly, if the rehosting assistant notices
%a divergence between the real and emulated executions, an SMT solver
%could be employed to find a device input that would cause the emulated
%execution to reach the sampled points. This latter strategy has been
%successfully used to by Jin and Orso~\cite{Jin2012} to reproduce
%failures seen in the field based on execution samples from the failing
%host.
%
%Finally, if it is known that the code of interest is active after the
%boot process has finished, the JTAG interface could be used to extract a
%snapshot of the memory and CPU state from the device, which might then
%be loaded into the emulator. Execution could then resume inside the
%emulator, allowing device driver code to be tested without going through
%the trouble of implementing the correct responses to the device
%initialization code. The system may, of course, still hang or crash due
%to missing device support, at which point the same strategies outlined
%earlier could be employed; however, supporting the post-boot behavior of
%the embedded peripherals (XXX: go back and use the word peripherals
%more) may prove to be a simpler task. Being able to skip ahead to the
%post-boot phase also has the advantage that it quickly allows an
%analysis to inspect the device code that would be most exposed during
%actual use.
%
