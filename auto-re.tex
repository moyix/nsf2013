\documentclass[letterpaper,twoside,11pt,headings=small]{scrartcl}

\PassOptionsToPackage{hang,small}{caption}
\usepackage[margin=1in]{geometry}
\usepackage{amsmath,amsfonts,amssymb,amsthm}
\usepackage{graphicx}
\usepackage[english]{babel}
\usepackage{subfig}
\usepackage{fancyvrb}
\usepackage{booktabs}
\usepackage{chappg}
\usepackage{xspace}
\usepackage{scrpage2}
\usepackage{pgf,tikz}
\usetikzlibrary{calc}
\usetikzlibrary{arrows}
\usepackage{ulem}
\normalem
\usepackage{fontspec}
\usepackage{xunicode}
\usepackage{xltxtra}
\defaultfontfeatures{Mapping=tex-text}
\setromanfont[Ligatures={TeX}]{Times New Roman}
\setsansfont[Ligatures={TeX}]{Helvetica Neue}
\setmonofont[Scale=MatchLowercase]{Menlo}
\usepackage{todonotes}
\usepackage{paralist}

\newcommand{\basetitle}{TWC: Medium: Automated Reverse Engineering of Commodity Software}
\newcommand{\thetitle}{\basetitle\xspace}
\newcommand{\dynamicsys}{\textsc{DynamicSystem}\xspace}
\newcommand{\challenge}[1]{\paragraph{Research Challenge:} \emph{#1}}

\pagestyle{scrheadings}
\clearscrheadfoot
\lohead{NSF 13-578 --- Secure and Trustworthy Cyberspace}
\rohead{\pagemark}
\lehead{\thetitle}
\rehead{\pagemark}
\setheadsepline{0.5pt}
\setfootsepline{0pt}
\setkomafont{pageheadfoot}{\sffamily\fontsize{9}{9}\selectfont}
\setkomafont{pagenumber}{\sffamily\fontsize{9}{9}\selectfont}

\usepackage[unicode,bookmarks,colorlinks,breaklinks,pdftitle={\basetitle},pdfauthor={}]{hyperref}
\hypersetup{linkcolor=black,citecolor=black,filecolor=black,urlcolor=black}

\begin{document}

\pagenumbering[B]{bychapter}

{\sffamily\bfseries
\begin{center}
\fontsize{16}{16}\selectfont Project Summary

\fontsize{13}{13}\selectfont \thetitle
\end{center}
\label{sec:summary}
}

Software, including common examples such as commercial applications or
embedded device firmware, is often delivered as closed-source binaries.  While
prior academic work has examined how to automatically discover vulnerabilities
in binary software, and even how to automatically craft exploits for these
vulnerabilities, answering basic security-relevant questions about
closed-source software remains difficult.  For instance, ideally one would like to
know whether software contains malicious functionality; whether it contains
obfuscated code or protocols indicative of malicious behavior; what security
properties does this software intend to provide; and, whether one can
recognize when software is deviating from normal, intended behavior at
runtime.

This project aims to provide algorithms and tools for answering these questions.
Leveraging prior work on emulator-based dynamic analyses, we propose techniques
for scaling this high-fidelity analysis to capture and extract whole-system
behavior in the context of embedded device firmware.  Using a combination of
dynamic execution traces collected from this analysis platform and binary code
analysis techniques, we propose techniques for automated structural analysis of
closed-source software, decomposing system and user-level programs into logical
modules through inference of high-level semantic behavior.  This decomposition
provides as output an automatically learned description of the interfaces and
information flows between each module at a sub-program granularity.

Using emulation-based dynamic analysis and structural decomposition as a
foundation, we will develop specific applications of this framework for
whole-system understanding.  As one application, we propose techniques for
automated detection and mitigation of software backdoors in embedded devices
such as SOHO routers, a recent example of which is the backdoor discovered in
the D-Link DIR-100 rev.~A broadband router~\cite{heffner:dlink-dir100}.

As a second application of our framework for whole-system understanding, we
propose techniques for automating the reverse engineering of encrypted network
protocols commonly used by malware to obfuscate their communication with
command-and-control servers.

\todo[inline]{Expand introduction.}

\paragraph{Intellectual Merit.} Reverse engineering is a critical capability
for understanding and responding to software-borne threats, and can enable
deep system understanding towards automated hardening of existing
closed-source platforms.  However, reverse engineering currently entails the training
of highly-skilled security professionals, as well as significant manual
effort, both processes that do not scale to meet the current needs of industry
and government.  The automated techniques and tools we will develop in the
course of this research will lay the groundwork for meeting this need, as well
as enable new automatic hardening and threat response capabilities.

\paragraph{Broader Impacts.} The research proposed herein will have a
significant impact outside of the security research community.  We will
incorporate the research findings of our program into our undergraduate and
graduate teaching curricula, as well as in extracurricular educational efforts
such as Capture-the-Flag that have broad outreach in the greater Boston and
Atlanta metropolitan areas.  The close ties to industry that the collective
PIs possess will facilitate transitioning the research into practical
defensive tools that can be deployed into real-world systems and networks. As
a result, the program will have broad impact on both the training of
next-generation cybersecurity professionals as well as the advancement of
defensive tool capabilities in operational environments.

\newpage
\pagenumbering[D]{bychapter}
\setcounter{page}{1}

{\sffamily\bfseries
\begin{center}
\fontsize{16}{16}\selectfont Project Description

\fontsize{13}{13}\selectfont \thetitle
\end{center}
}

\section{Research Overview}
\label{sec:overview}

\subsection{Project Goals and Scope}
\label{sec:overview:goals}

\subsection{Embedded Firmware Emulation}
\label{sec:overview:firmware}

\challenge{CHALLENGE}

\challenge{CHALLENGE}

\challenge{CHALLENGE}

\subsection{Structural Decomposition}
\label{sec:overview:structure}

\challenge{CHALLENGE}

\challenge{CHALLENGE}

\challenge{CHALLENGE}

\subsection{Mitigating Software Backdoors}
\label{sec:overview:backdoors}

\challenge{CHALLENGE}

\challenge{CHALLENGE}

\challenge{CHALLENGE}

\subsection{Encrypted Protocol Fuzzing}
\label{sec:overview:fuzzing}

\challenge{CHALLENGE}

\challenge{CHALLENGE}

\challenge{CHALLENGE}

\subsection{Outcomes and Deliverables}
\label{sec:overview:outcomes}

\section{State of the Art and Previous Work}
\label{sec:related}

\section{Embedded Firmware Emulation}
\label{sec:research:firmware}

The devices that surround us every day, from cars to coffee makers,
contain complex software that has received little public examination.
Because these devices are increasingly being connected to the Internet,
it is vital that their security be rigorously evaluated.

To date, the analysis of embedded firmware has required source
code~\cite{davidson:2013:fie}, used only static
analysis~\cite{dreissen:2012:satphone}, or applied only to Linux-based
firmware. Many advanced analyses (e.g., concolic
execution~\cite{godefroid:2005:dart}, dynamic taint analysis, and
fuzzing) require dynamic analysis. Embedded devices run a wide variety
of operating systems (Windows CE, QNX, VxWorks, and Cisco IOS, to name
but a few), and so to have broad impact we must find analyses that do
not depend on a single operating system.

To effectively support sophisticated dynamic analyses for real embedded
systems, we need a platform that can run embedded firmware in a
whole-system emulator such as \dynamicsys, allowing full visibility
into all code running on the system. Although emulators such as QEMU
support a large number of processors such as ARM, MIPS, and PowerPC, a
given embedded device firmware will typically not run due to missing
support for peripheral devices.

Missing peripheral support poses an especially severe problem from a security
standpoint. Because peripherals are the means by which embedded systems
interact with the outside world, an analysis tool that cannot inspect the
device drivers for the platform will miss most of the attack surface. In order
to close this gap, we propose to develop automated techniques for building
models of embedded peripherals that can then be used by \dynamicsys to run
embedded firmware.

\subsection{Assumptions and Prerequisites}

Our proposed work makes the following assumptions:%
\begin{inparaenum}[\itshape a\upshape)]
    \item it is possible to obtain a copy of the target firmware, e.g.,
    by extracting it from flash memory or via the manufacturer's update
    tool;
    \item the CPU architecture on which the firmware runs can be
    determined, and the target emulator supports this CPU architecture;
    and,
    \item the load address and entry point of the firmware in memory can
    be found.
\end{inparaenum}

We believe that these assumptions are reasonable for many consumer
devices, and in any case solving these problems is orthogonal to the
main task of firmware emulation.

\subsection{Technical Approach: Software-Guided Emulation}

Our key insight is that \emph{firmware itself contains a rough
specification of the hardware on which it runs}: the checks and
operations performed on hardware-supplied values can be interpreted as
constraints on those same values. Thus, when a poorly emulated piece of
hardware misbehaves, the way the firmware acts in response can to
some extent be used as an oracle for the correct behavior.

To make this idea more concrete, consider an example where firmware
reads a value from memory-mapped device memory and checks whether its
value is less than \texttt{0x100}. If value returned by emulated
hardware is greater than this value, the firmware may print an error
message and halt. Upon detecting this error, our system could then infer
the constraint and retry the emulation with a response that is in range
and allows execution to continue.

However, as this example makes clear, we need a way to detect when an
emulated execution has gone astray. Also, iteratively finding
conditionals in the firmware that relate to hardware constraints may be
intractable computationally, as the search space is potentially
exponential.

To solve these problems, we propose to develop ways of inferring the
actual device's execution path through the firmware image. For example,
if the device has a serial port that produces debug messages, these
strings can be matched within the firmware binary find the code that
emits them. Thus, one can use a sequence of debugging messages as a
``trail of breadcrumbs'' to infer a valid path through the firmware
image. This means that the symbolic execution can likewise follow this
path through the firmware, collecting constraints on the device inputs
along the way.

Additional research challenges remain, however. Although the process
described above will work for simple peripherals, devices that keep
complex state or that do computation of their own outside the CPU will
be difficult to handle. In addition, the use of interrupts and direct
memory access (DMA) by peripherals is not handled by this approach, and
new advances are needed. We hope that by overcoming these research
challenges we can arrive at a system which automatically generates
reasonable device models for many peripherals, reducing the amount of
manual reverse engineering required to a minimum.

\subsection{Preliminary Results: Modeling Simple Devices}

We have performed some initial investigation into this problem using the
firmware of an ARM-based HP LaserJet 4050dn printer. On the device, we
were able to locate a serial port, which provided a stream of boot
messages to act as the ``breadcrumbs'' described in the previous
section. We were also able to extract the firmware image from the
devices and load it in \dynamicsys.

By combining \dynamicsys with a symbolic execution engine and SAT
solver, we were able to \emph{automatically} infer peripheral device
models for many of the devices used in the early stages of boot,
including the serial port and NVRAM configuration. Our automatically
generated device models allow the system to partially boot, reproducing
the real serial console output, until it reaches code that stops the
system, waiting for an interrupt.

For comparison, before trying to automatically generate peripheral
device models we also attempted to create them by hand, a process which
took several weeks to reach a similar state --- a sharp contrast with
the few minutes of computation required for our automated system.

These results indicate that the approach described in the previous
section has the potential for reducing the work required to emulate an
embedded firmware, but many interesting research problems are still
unsolved.

%For example, one could use JTAG to create a device snapshot that
%could be revived in QEMU, allowing the analysis described above to skip
%over potentially complex device initialization code and more quickly get
%to the business of testing the attack surface of the device.
%Alternatively, JTAG could be used to sample the device's execution,
%allowing direct comparison with the emulated version; this ``trail of
%breadcrumbs'' could even be used as input to a constraint solver that to
%identify what device inputs are needed to reach the sampled code
%paths.\footnote{This last approach draws on techniques recently proposed
%by Jin and Orso~\cite{Jin2012} to reproduce software failures seen in
%the field.}

%. Many embedded devices
%support IEEE 1149.1, the Standard Test Access Port and Boundary-Scan
%Architecture~\cite{jtag}, more commonly known as JTAG. The JTAG
%interface of many embedded devices offers some amount of debugging
%support, and allows one to halt the processor, read memory and CPU
%state, and so on. Such information could be used by an automated
%rehosting solution in a number of ways, as we detail below.
%
%First and most trivially, the ability to peek into the runtime state of
%a correctly functioning device would allow features such as the location
%and contents of the firmware to be determined directly, which would make
%the initial phase of the procedure described above more efficient.
%
%The JTAG interface could also be used to sample the execution of the
%real device during boot. This would provide a ``trail of breadcrumbs''
%that the emulator could then follow. When the boot fails in the
%emulator, these sample points could guide the search for a solution.
%For example, if the emulated boot hangs in a loop while waiting for an
%interrupt from a non-existent device, the execution samples might
%indicate that the next code to be executed lies in an interrupt handler,
%which would allow the rehosting assistant to create a stub that asserts
%the appropriate interrupt. Similarly, if the rehosting assistant notices
%a divergence between the real and emulated executions, an SMT solver
%could be employed to find a device input that would cause the emulated
%execution to reach the sampled points. This latter strategy has been
%successfully used to by Jin and Orso~\cite{Jin2012} to reproduce
%failures seen in the field based on execution samples from the failing
%host.
%
%Finally, if it is known that the code of interest is active after the
%boot process has finished, the JTAG interface could be used to extract a
%snapshot of the memory and CPU state from the device, which might then
%be loaded into the emulator. Execution could then resume inside the
%emulator, allowing device driver code to be tested without going through
%the trouble of implementing the correct responses to the device
%initialization code. The system may, of course, still hang or crash due
%to missing device support, at which point the same strategies outlined
%earlier could be employed; however, supporting the post-boot behavior of
%the embedded peripherals (XXX: go back and use the word peripherals
%more) may prove to be a simpler task. Being able to skip ahead to the
%post-boot phase also has the advantage that it quickly allows an
%analysis to inspect the device code that would be most exposed during
%actual use.
%


\section{Structural Decomposition}
\label{sec:research:structure}

\cite{csallner:icse2008:dysy}
\cite{krka:icsa2010:inference}
\cite{chipounov:asplos2011:s2e}

% \begin{itemize}
%     \item Analysis phase
%         \item Recover module boundaries
%         \item Learn high-level semantic behavior from execution traces
%         \item Recover information flows between modules
%     \item Rewriting phase
%     \begin{itemize}
%         \item Detect deviations from learned invariants
%         \item Harden interfaces between modules
%     \end{itemize}
% \end{itemize}

When analyzing software, it is often uncommon to have access to the source
code of the program of interest.  Manually finding vulnerabilities or
malicious behavior in binary programs is time-consuming, error-prone, and
typically requires allocating highly-experienced reverse engineers to the
task. Therefore, it is desirable to have precise and scalable program analyses
of native binary executables to discover vulnerabilities or malicious
components of these programs, render benign applications more resilient to
known classes of attack, and to deep insight into a system's runtime state.
Systems like \dynamicsys provide a solid basis for developing analyses of this
sort.

Given the capability to perform precise static and dynamic data-flow analysis
on binaries, it becomes possible to answer higher-level questions such as:
\begin{inparaenum}[i)]
    \item what are the distinct modules that comprise the program or system,
    \item what are the interfaces that these modules expose to their environment,
    \item what are the information flows between these modules, and
    \item can we characterize the normal behavior of a system in terms of this higher
        level of abstraction and detect deviations from expected behavior?
\end{inparaenum}

We propose the development of an advanced, automated binary program analysis
platform that aims to automatically answer these questions.  Specifically, our
platform leverages \dynamicsys to identify the modules that comprise a set of
binary programs, the interfaces they expose, and characterize their structure
and expected behavior.  The platform leverages \dynamicsys and uses a combination
of static and dynamic analyses to automatically infer the high-level structure
and interfaces of large-scale programs for which only the binary -- and the
ability to execute it -- is given.  A precise and efficient static analysis
allows the platform to construct reliable control-flow graphs (CFGs) of the
programs under test.  Whole-system emulation is then used to perform
fine-grained, instruction-level, dynamic analysis.  The resulting execution
traces allow the platform to perform advanced data-flow analyses that refine
the statically-derived CFG.  Using these program analyses, we will then
extract behavioral models of the program under test that can be compared to a
\emph{minimal} specification of its expected behavior.  This capability will
allow our platform to automatically decompose a set of binary executables into
their constituent logical modules, as well as identify both direct and
indirect communication channels with other modules and the external
environment.

The models extracted from a binary program under analysis will then be used as
input to a runtime monitoring component.  This monitor will periodically
compare the execution state of the program to invariants encoded in the model,
allowing for the efficient detection of abnormal deviations from expected
behavior.  Additionally, the higher-level characterization of program behavior
learned in the previous phase will allow for better explanatory power in the
reports generated by the monitor, leading to increased insight into whole
system behavior on the part of system operators.

The envisioned architecture of our analysis platform is depicted in
Fig.~\ref{fig:decomposition-arch}.  The approach consists of two logical
phases.  In the first, the analysis components are applied to a replica of a
deployed system within an isolated, emulated environment.  From these
analyses, a model of high-level program structure and legitimate behavior is
extracted.  This model is then used in the second phase, where a secure
runtime component checks the execution of the deployed system against the
model.  Any deviations from this model are indicated in the form of high-level
alerts to system operators.

In the following, we elaborate upon each of the components of our system for
system taint analysis and monitoring.

\begin{figure}[t]
    \centering
    \missingfigure{Architecture of structural decomposition platform.}
    \caption{Envisioned architecture of the analysis platform.
    A combination of static and dynamic analyses are performed on a replica
    of a deployed system within an isolated, emulated environment.
    The analyses extract a model of high-level program structure and
    legitimate behavior that is then used to monitor the state of the
    deployed system.}
    \label{fig:decomposition-arch}
\end{figure}

\subsection{Dynamic Analysis Engine}

The dynamic analysis engine performs its analysis inside of a high-fidelity
emulated environment based on the \dynamicsys system.  The system monitors the
activity of programs under test during runtime, and has full visibility into
the state of the entire machine at an instruction-level granularity.  This
precision complements the strengths of static analyses; since dynamic analysis
operates over concrete values, the technique has the significant advantage
that it can easily handle obfuscated, self-modifying, and concurrent code.  We
refer the reader to the discussion on \dynamicsys in
Section~\ref{sec:research:firmware} for more details.

In addition to operation over concrete values, we additionally incorporate
dynamic symbolic execution over these traces.  Dynamic symbolic execution
substitutes symbolic values for inputs to individual modules, allowing for
the identification of new inputs that can be provided to the program under
test in order to explore previously uncovered code.  This capability relies
upon the identification of module boundaries and interfaces, which we describe
in Section~\ref{sec:research:structure:modules}.

The output of the dynamic analysis engine is a set of execution traces
augmented with data flow information.  These traces completely describe the
evolution of the state of the program under test with respect to the test
inputs, its interaction with the environment, and the propagation of data
through the system.

\subsection{Static Analysis Engine}

The static analysis engine provides a precise and scalable set of analyses
that operate directly on binary program executables, without the need for
source code.  Incorporating a robust static analysis component allows our
platform to achieve high coverage of program behaviors over possible inputs
without having to actually exercise the program under test on those inputs.
However, static analysis has well-known deficiencies -- especially with
respect to binary executables -- that are important to address.

To that end, we will leverage our prior work on static binary analysis. In
doing so, our platform will be based upon a strong foundation of analyses that
can handle imprecision at multiple levels, including both disassembly and
control flow.

\paragraph{Robust disassembly.} A well-known challenge for binary static
analysis is the difficulty of achieving full coverage of the code contained in
an executable image. Our prior work has studied techniques for improving the
robustness of binary disassembly, and we intend to leverage both symbolic
execution and statistical analysis to that end~\cite{kruegel:sec2004:disasm}.
The use of a symbolic execution engine allows our disassembler to handle many
computed jumps that would otherwise be difficult to resolve statically.

Our work also incorporates statistical techniques to probabilistically
identify likely code regions in binary executables.  An example of this is to
collect, \emph{a priori}, digram probabilities for pairs of instructions from
a corpus of known benign programs.  Then, this can be used during disassembly
to probabilistically identify code regions, or reduce the imprecision
introduced by a (overly-conservative) widened symbolic jump target.

Malicious code also often utilizes obfuscation techniques to hinder static
analysis, such as overlapping instruction sequences as seen in variable length
ISAs such as ix86, or switching between multiple supported ISAs as in the case
of ARM and the Thumb family of instruction sets.  We will develop techniques
to handle these classes of obfuscation.

\paragraph{Control-flow subgraph matching.} One capability that is
particularly useful in several contexts relating to binary static analysis is
fuzzy control-flow subgraph matching.  Our prior work has demonstrated its
utility when performing efficient online detection of polymorphic worm
propagation on the network~\cite{kruegel:raid2005:worm}. We anticipate that
this capability will be useful in achieving the goal of module identification
and behavioral characterization.

Our technique for fuzzy control-flow subgraph matching takes two binary CFGs
as input, where graph nodes and edges represent basic blocks and control flow
transfers, respectively.  The algorithm colors the graph based on the
semantics of instructions contained in each basic block, extracts
$k$-subgraphs for small values of $k$ by computing a spanning tree, and then
performs fast subgraph matching to determine the overlap between $k$-subgraphs
from each CFG. By applying abstraction and decomposing the CFGs into
subgraphs, our matching technique is able to efficiently discover
semantically-similar CFGs.  We anticipate that this matching technique can
serve as a robust base upon which to enable module identification.

\subsection{Analysis Capabilities}
\label{sec:research:structure:modules}

The combination of our proposed static and dynamic analyses for binary
programs will enable several key capabilities: automated module
identification, behavioral characterization, interface enumeration, and
runtime monitoring.

\paragraph{Module identification and characterization.} Our composition of
program analyses will allow our platform to automatically identify the
constituent modules of a binary set of programs under test and, in conjunction
with limited \emph{a priori} domain knowledge of the program, classify each
module according to its function.  Module identification will proceed by
iterating between two distinct phases. First, a robust static disassembly and
context-sensitive control-flow analysis will allow our platform to obtain an
initial control-flow graph (CFG) of the program. During this phase,
domain-specific knowledge can be leveraged to statically classify subgraphs of the
CFG as belonging to particular modules -- e.g., by recognizing the definition
and use of interrupt vectors specific to a particular machine architecture.

During the second phase, this initial static model will be refined by
monitoring the runtime behavior of the program under test.  Ambiguities
resulting from natural limits on the precision of static analysis can be
resolved by observing program behavior over concrete inputs and applying
dynamic symbolic execution over the resulting traces.  For instance,
incomplete coverage of the binary program under test can be improved by
resolving the targets of statically unknown indirect control flow transfers
during the dynamic analysis phase.  The dynamic trace analysis will be
combined with judicious application of novel program invariant inference
techniques~\cite{ernst:2009:daikon,csallner:icse2008:dysy,krka:icsa2010:inference}.
By associating code structure with state invariants at critical program
points, our platform will be capable of recognizing many common program design
patterns of interest.

Iterating between these two phases will allow the platform to identify control
flow patterns that correspond to high-level specifications of expected
behavior defined \emph{a priori} for one or more modules.  For instance, in
the case where a multitasking real-time operating system (RTOS) is being
analyzed, one would expect to observe a control transfer as a result of a
timer interrupt, followed by a deterministic loop over an array or list of
process descriptors, followed by the resumption of a selected thread of
control.  Our platform will allow for the specification and matching of such
high-level patterns of control flows against real binaries and execution
traces.  In the case that concurrent execution of multiple modules is
encountered by the platform, domain-specific knowledge will be leveraged for
the particular machine architecture to isolate and extract traces for each
distinct thread of execution.

\paragraph{Invariant analysis.} Prior work on invariant analysis major
limitations that we aim to improve in our proposed work.  First, current
invariant invariant approaches are primarily limited to either predefined
templates~\cite{ernst:2009:daikon} or extraction of low-level invariants from
program code~\cite{csallner:icse2008:dysy}.  While extraction of low-level
invariants is a promising direction, we will develop techniques for inferring
high-level invariants over module behavior and inter-module communication in
the context of high-level module behavior summaries that are security-relevant
-- i.e., useful for enforcing security properties such as isolation between
modules with different privilege levels or belonging to distinct security
principals.

A second limitation of current invariant detection approaches is that they
require either source code, bytecode, or binary programs compiled with
debugging information.  In our work, we aim to extend invariant analysis to
cover binary programs for which we do not have access to such information.
This will require differential analysis of the memory of the program during
execution at selected program points.  In addition, we will define algorithms
to scope the differential program state analysis to the subset of memory that
can be accessed by the current region of code -- e.g., the currently executing
function.  Determining this memory subset will require incorporation of both
static and dynamic information.

Finally, we aim to address the restriction that current invariant analyses are
primarily limited to the form of pre- and post-conditions on function or
method invocations.  A major part of our work will include the identification
and application of invariants in a fine-grained manner, where we will select
critical points in the CFG using structural analysis to apply our invariants.
As an example, using this approach we aim to discover program points such as
loops over critical data structures, or switch tables that correspond to
individual protocol state handlers

\paragraph{Module interface enumeration.} Once a set of modules has been
identified from the combination of the statically-derived CFG and runtime
monitoring, the analysis platform will proceed to automatically identify the
interfaces exposed by each module.  This step will take into account
domain-specific knowledge regarding machine architecture -- e.g., to identify the use
of special instructions that allow for control transfers between modules such
as system call service invocations.  To enable richer semantic understanding
of not only the vectors for entering a module, but also the types of data that
accompanies these control transfers, the platform will leverage a combination
of static and dynamic data-flow analyses to characterize the number and types
of parameters for control transfers across modules.  In cases where memory is
shared between modules, points-to analyses will be leveraged to characterize
the data shared between modules and the sets of modules that contain
references to particular memory objects.  For instance, returning to the
example of an RTOS, our platform will be able to identify well-known
structures such as process lists or memory descriptors as channels of
information flow that serve as an \emph{implicit} interface between
logically-distinct modules.

\paragraph{Runtime monitoring.} The result of the previous phase will be a
high-level structural model of the system under test in the form of a
decomposition of independent models, a characterization of their function and
behavior, and execution state invariants at critical program points.  This
model will then be used by a runtime component that will monitor the execution
of the deployed system. The monitoring component will use dynamic state
introspection and will be protected from attacks against integrity using a
robust isolation mechanism. Introspection will occur at carefully selected
program points in order to balance behavioral coverage with performance
requirements.

Deviations from the model that are indicative of anomalous and potentially
malicious factors will be detected and reported.  Additionally, due to the
characterization analysis performed during model construction, reports will
contain high-level information concerning \emph{why} the observed behavior is
anomalous, leading to increased whole-system insight on the part of system
analysts and operators.

\section{Summarizing Software Execution}
\label{sec:research:autosummary}

Existing work on reverse engineering binary code has typically focused
on decompilation~\cite{schwartz:2013:decomp,cifuentes:1995:decomp}.
However, for quickly understanding the behavior and capabilities of a
program, source code may not be the most efficient representation.
Instead, one may desire \emph{semantically meaningful summaries} of
parts of a program or execution trace. For example, portions of an
execution might be summarized as ``argument parsing'',
``serialization'', or ``encryption''.

This kind of high-level labeling is often performed implicitly by human
reverse engineers as they analyze a program. One may glance over the
functions in a binary to get a rough idea of their purpose before
deciding where to spend one's (limited) analysis time. We propose to
build a system that can quickly provide these kinds of high-level
semantic summaries \emph{automatically}. 

To accomplish this, we propose to look to the area of streaming
analytics. Previous work from Georgia Tech~\cite{dolangavitt:2013:tzb}
has demonstrated that viewing a program execution as a large number of
streaming memory accesses, combined with an analysis of the contents of
these streams, can be a powerful tool for program explication. This
notion can be extended beyond memory accesses to other artifacts of
computation, effectively treating an execution as a generator for
streams of observations of the program's state. By collecting statistics
on these streams and comparing them to a corpus of labeled exemplars, we
will be able to classify execution fragments according to high-level
semantic descriptions.

\subsection{Programs as Streaming Data}

To perform a high-level labeling, we need to generate a set of
observations about the behavior of portions of program execution both in
the training corpus and for new executions we wish to understand. In
previous work~\cite{dolangavitt:2013:tzb}, this took the form of streams
of data from memory accesses. In our proposed system, we will augment
these with a number of other observables, for example:

\begin{itemize}
    \item System, API, and function calls
    \item CPU register contents over time
    \item Assembly instruction mnemonics in a sliding window over the
    instruction stream
    \item Externally visible outputs such as disk, network, and IPC
\end{itemize}

Each of these can be combined with its context within the program (i.e.,
the module as identified by our structural decomposition and the calling
context) to break up these data streams. The key idea is that these
sub-streams will now contain data that is of the same semantic type, as
each is generated from within the same part of the program. Taken
together, the content of these streams should thus form a fingerprint
for the code's functionality. Some streams may be noisier than others,
but creating robust classifiers from noisy signals is a well-studied
problem in machine learning.

\subsection{Creating a Labeled Corpus}

A natural approach to creating a labeled corpus of program functionality
is to manually examine many programs and assign meaningful labels to
their various components. Aside from the daunting amount of work this
would require to build a reasonably-sized corpus, this strategy is also
inherently underspecified. Any labeling effort would need to decide
\emph{a priori} on a consistent set of labels, but the diversity of
real-world program behaviors means any such label set would likely be
incomplete.

Instead we propose to generate our corpus from precisely the semantic
information that is usually removed during compilation: comments and
variable names. Naturally, these labels will not be as precise as those
a human would create, but in aggregate they should provide valuable
information on the meaning of the code in question.

Thus, to create a corpus one can start by compiling a large number of
open source programs with debugging information. The resulting binary
code can then be given labels derived from the comments and variable
names. Finally, the programs can then be run in \dynamicsys to create
the streams of observations that will be used in classification.

\subsection{Classifying Execution Fragments}

Given a new execution, we can now generate the same streams of
observations as for the exemplar corpus. These observations can then be
matched against the models generated in training to find those that most
closely match. The output is a labeled instruction stream that assigns
\emph{meaning} to execution fragments.

This problem can also be seen as a statistical machine translation
problem with three parallel corpora: the comments and variable names in
the code, the binary code produced from compilation, and the streams of
observations produced by running the code. The first two are explicitly
related via the source code to binary mapping produced by compilation.
The latter two can be seen as implicitly related by some noisy channel. 
Using techniques from machine translation, we can find a mapping between
the first and third, giving us the desired semantic labeling.


\section{Encrypted Protocol Fuzzing}
\label{sec:research:fuzzing}

Fuzz testing or fuzzing has been widely used in software vulnerability
discovery. The basic idea behind fuzzing is to generate a large number
of malformed inputs by either mutating normal inputs or directly
constructing them according to predefined generation rules, and then use
the malformed inputs to test the target software. Despite the simplicity of
the concept, fuzzing has proven very effective in uncovering previously
unknown vulnerabilities in complex software. However, existing fuzzing
technologies are hardly applied to undocumented and encrypted protocols,
because the generated inputs do not satisfy integrity validation checks or
become meaningless after being decrypted and are dropped.

For example, many instant messengers (IM) encrypt all network
traffic for privacy protection. In this case, upon receiving a message,
such an IM client would verify the integrity of the message, decrypt the
payload of the message, and further handle the decrypted data.  In
comparison with the code responsible for integrity verification and
message decryption, the code responsible for handling decrypted data is
usually more vulnerable. Unfortunately, without knowing the IM protocol,
network-fuzzing tools cannot construct a meaningful message that can
reach the most vulnerable region.

We propose to develop a smart fuzzing platform based on the
\dynamicsys system and the module identification analysis, with the goal
of coping with encrypted network protocols and improving the effectiveness
of fuzzing.  Our intuition is that in order to support
bidirectional communication, the programs (such as IM clients) capable
of decrypting messages usually own the encryption capability at the same
time.  By using dynamic data flow tracking and the module identification
analysis, we expect to identify the modules in a program that are
responsible for message encryption and decryption. As a result, rather
than constructing a malformed network packet from scratch, we
propose to leverage the program themselves to build syntactically
correct but semantically incorrect data and then use the data to test
the programs.

Consider the case of IM clients as a concrete example.  After two IM clients
initialize a communication channel, we can introduce data mutation to the
input buffer of the encryption function at runtime on one side and then send
the encrypted message to the other side. In this case, the encrypted message
would be completely accepted by the other side and reach a region of code that
is more likely to be vulnerable.  Furthermore, during the process of how the
IM client handles a correctly decrypted message, we can apply the dynamic data
flow tracking technique to identify the input bytes that flow into security
sensitive operations, meaning they are more likely to trigger vulnerabilities.
Then we use this knowledge to guide the data mutation.


\section{Broader Impact and Educational Outreach}
\label{sec:impact}

\subsection{Curriculum Development}
\label{sec:impact:curriculum}

\subsection{Education and Outreach}
\label{sec:impact:education}

\subsection{Technology Transfer and Dissemination}
\label{sec:impact:tech-transfer}

\section{Project Time Plan}
\label{sec:time-plan}

\paragraph{Year 1.}

\paragraph{Year 2.}

\paragraph{Year 3.}

\section{Qualifications of the PIs and Previous NSF Support}
\label{sec:qualifications}

\newpage
\pagenumbering[E]{bychapter}
\setcounter{page}{1}
\bibliographystyle{acm}
\bibliography{satc}

\newpage
\pagenumbering[J]{bychapter}
\setcounter{page}{1}
\setcounter{section}{0}

\end{document}
