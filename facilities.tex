\documentclass[letterpaper,11pt]{scrartcl}

\PassOptionsToPackage{hang,small}{caption}

\usepackage[margin=1in]{geometry}
\usepackage{amsmath,amsfonts,amssymb,amsthm}
\usepackage{graphicx}
\usepackage[english]{babel}
\usepackage{subfig}
\usepackage{booktabs}
\usepackage{fancyvrb}
\usepackage{xspace}
\usepackage{scrpage2}
\usepackage{pgf,tikz}
\usetikzlibrary{calc}
\usetikzlibrary{arrows}
\usepackage{ulem}
\normalem
\usepackage{fontspec}
\usepackage{xunicode}
\usepackage{xltxtra}
\defaultfontfeatures{Mapping=tex-text}
\setromanfont[Ligatures={TeX}]{Times New Roman}
\setsansfont[Ligatures={TeX}]{Helvetica Neue}
\setmonofont[Scale=MatchLowercase]{Menlo}

\usepackage[unicode,bookmarks,colorlinks,breaklinks,pdftitle={},pdfauthor={}]{hyperref}
\hypersetup{linkcolor=black,citecolor=black,filecolor=black,urlcolor=black}

\newcommand{\thetitle}{TWC: Medium: Collaborative: Automated Reverse Engineering of Commodity Software\xspace}

\pagestyle{scrheadings}
\clearscrheadfoot
\lohead{\thetitle}
\rohead{\pagemark}
\lehead{\thetitle}
\rehead{\pagemark}
\setheadsepline{0.5pt}
\setfootsepline{0pt}
\setkomafont{pageheadfoot}{\sffamily\fontsize{9}{9}\selectfont}
\setkomafont{pagenumber}{\sffamily\fontsize{9}{9}\selectfont}

\begin{document}

% \pagenumbering[H]{bychapter}

\title{Facilities, Equipment, and Other Resources}
\subtitle{\thetitle}
\author{}
\date{}
\maketitle

\section*{Northeastern University}

Northeastern University is a private research university in Boston and a
leader in integrating classroom learning with real-world experience.
Northeastern has eight colleges and offers undergraduate majors in 65
departments. At the graduate level, the university offers more than 125
programs and awards masters, doctoral, and professional degrees.
Northeastern's 73-acre award winning campus is the home of more than 35
specialized research and education centers.  Northeastern has significant
depth in terms of faculty and research strengths in the areas of systems and
systems security. In 2010, Northeastern University received a \$12 million
gift from one of its alumni, George J.~Kostas, E'43, H'07, to build a secure,
state-of-the-art homeland security research facility on the university's
Burlington campus. The new George J.~Kostas Research Institute for Homeland
Security, a multi-story building, has been designed in accordance with US
government standards for secure facilities and gives Northeastern the capacity
and clearances to conduct secure research in a restricted environment. Within
the building, sensitive, interdisciplinary research takes place in areas
critical to national security, including cryptography, data security, systems
security, information assurance, detection of explosives, and energy
harvesting.  The College of Engineering and the College of Computer and
Information Science also have programs in cyber infrastructure protection,
networks, algorithms, and languages.

The environment at Northeastern University offers a unique opportunity for the
PIs to carry out this project.  The proposed research directions complement
well the existing strengths in multiple areas of computer science, including
networks (Guevara Noubir, Alan Mislove, Christo Wilson, and David Choffnes),
systems (Peter Desnoyers and Gene Cooperman), algorithms (Rajmohan Rajaraman),
formal methods (Pete Manolios and Thomas Wahl), and programming languages
(Matthias Felleisen, Mitch Wand, Olin Shivers).

% \section*{Georgia Institute of Technology}

% The Georgia Tech Information Security Center (GTISC) is located in the Klaus
% Advanced Computing Building and is comprised of the Information Security Lab
% and the Voice Over IP (VOIP) Lab.  GTISC operates a substantial number of
% network, computational server and storage resources  to support its research
% activities in the area information security.

% Georgia Tech's state-of-the-art network provides capabilities with few
% parallels in academia or industry, delivering unique and sustained competitive
% advantage to Georgia Tech faculty, students, and staff. Since the mid-80’s
% Georgia Tech and OIT have provided instrumental leadership in high-performance
% networks for research and education (R\&E) regionally, nationally, and
% internationally.  A founding member of Internet2 (I2) and National LambdaRail
% (NLR) – high bandwidth networks dedicated to the needs of the research and
% education community – Georgia Tech manages and operates Southern Crossroads
% (SoX, the I2 regional GigaPOP) and Southern Light Rail (SLR, the NLR regional
% aggregation). We work within six Southeastern states to make affordable high-
% performance network access and network services available to researchers and
% faculty at Georgia Tech, their collaborators, other higher-education systems,
% K-12 systems, and beyond.  Georgia Tech's network has high-performance
% connectivity to other members of the research and education community
% world-wide through dual 10 gbps (gigabits per second) links to SoX/SLR, which
% has peerings with NLR Packetnet, Internet2 Network, TransitRail, Oak Ridge
% National Labs (ORNL), the Department of Energy's Energy Sciences Network
% (ESNet), NCREN, NASA's NREN, MREN, FLR, Peachnet, LONI, 3ROX, as well as other
% SoX participants in the Southeast.  In addition to the exceptional R\&E
% network connectivity provided to all Georgia Tech faculty, students, and
% staff, dedicated bandwidth in support of specific collaborations and research
% is also possible.

\end{document}
