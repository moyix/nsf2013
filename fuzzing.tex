Fuzz testing or fuzzing has been widely used in software vulnerability
discovery. The basic idea behind fuzzing is to generate a large number
of malformed inputs by either mutating normal inputs or directly
constructing them according to predefined generation rules, and then use
the malformed inputs to test the target software. Despite the simplicity of
the concept, fuzzing has proven very effective in uncovering previously
unknown vulnerabilities in complex software. However, existing fuzzing
technologies are hardly applied to undocumented and encrypted protocols,
because the generated inputs do not satisfy integrity validation checks or
become meaningless after being decrypted and are dropped.

For example, many instant messengers (IM) encrypt all network
traffic for privacy protection. In this case, upon receiving a message,
such an IM client would verify the integrity of the message, decrypt the
payload of the message, and further handle the decrypted data.  In
comparison with the code responsible for integrity verification and
message decryption, the code responsible for handling decrypted data is
usually more vulnerable. Unfortunately, without knowing the IM protocol,
network-fuzzing tools cannot construct a meaningful message that can
reach the most vulnerable region.

We propose to develop a smart fuzzing platform based on the
\dynamicsys system and the module identification analysis, with the goal
of coping with encrypted network protocols and improving the effectiveness
of fuzzing.  Our intuition is that in order to support
bidirectional communication, the programs (such as IM clients) capable
of decrypting messages usually own the encryption capability at the same
time.  By using dynamic data flow tracking and the module identification
analysis, we expect to identify the modules in a program that are
responsible for message encryption and decryption. As a result, rather
than constructing a malformed network packet from scratch, we
propose to leverage the program themselves to build syntactically
correct but semantically incorrect data and then use the data to test
the programs.

Consider the case of IM clients as a concrete example.  After two IM clients
initialize a communication channel, we can introduce data mutation to the
input buffer of the encryption function at runtime on one side and then send
the encrypted message to the other side. In this case, the encrypted message
would be completely accepted by the other side and reach a region of code that
is more likely to be vulnerable.  Furthermore, during the process of how the
IM client handles a correctly decrypted message, we can apply the dynamic data
flow tracking technique to identify the input bytes that flow into security
sensitive operations, meaning they are more likely to trigger vulnerabilities.
Then we use this knowledge to guide the data mutation.
