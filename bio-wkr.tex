\documentclass[letterpaper,11pt]{scrartcl}

\PassOptionsToPackage{hang,small}{caption}

\usepackage[margin=1in]{geometry}
\usepackage{amsmath,amsfonts,amssymb,amsthm}
\usepackage{graphicx}
\usepackage[english]{babel}
\usepackage{subfig}
\usepackage{booktabs}
\usepackage{tabularx}
\usepackage{fancyvrb}
\usepackage{xspace}
\usepackage{scrpage2}
\usepackage{pgf,tikz}
\usetikzlibrary{calc}
\usetikzlibrary{arrows}
\usepackage{ulem}
\normalem
\usepackage{fontspec}
\usepackage{xunicode}
\usepackage{xltxtra}
\defaultfontfeatures{Mapping=tex-text}
\setromanfont[Ligatures={TeX}]{Times New Roman}
\setsansfont[Ligatures={TeX}]{Helvetica Neue}
\setmonofont[Scale=MatchLowercase]{Menlo}

\usepackage[unicode,bookmarks,colorlinks,breaklinks,pdftitle={},pdfauthor={}]{hyperref}
\hypersetup{linkcolor=black,citecolor=black,filecolor=black,urlcolor=black}

\pagestyle{scrheadings}
\clearscrheadfoot
\lohead{\thetitle}
\rohead{\pagemark}
\lehead{\thetitle}
\rehead{\pagemark}
\setheadsepline{0.5pt}
\setfootsepline{0pt}
\setkomafont{pageheadfoot}{\sffamily\fontsize{9}{9}\selectfont}
\setkomafont{pagenumber}{\sffamily\fontsize{9}{9}\selectfont}

\newcommand{\thetitle}{TWC: Medium: Automated Reverse Engineering of Commodity Software\xspace}

\begin{document}

\title{Biographical Sketch}
\subtitle{William Robertson}
\author{}
\date{}
\maketitle

\subsection*{Professional Preparation}

\begin{tabularx}{\textwidth}{XXc}
\toprule
\sffamily\textbf{Institution} & \sffamily\textbf{Degree/Area} &
    \sffamily\textbf{Year(s)}\\
\midrule
UC Santa Barbara & B.S., Computer Science & 1997--2002\\
UC Santa Barbara & Ph.D., Computer Science & 2003--2009\\
UC Berkeley & Postdoctoral Researcher & 2009--2011\\
\bottomrule
\end{tabularx}

\subsection*{Appointments}

\begin{tabularx}{\textwidth}{XXc}
\toprule
\sffamily\textbf{Institution} & \sffamily\textbf{Position} &
    \sffamily\textbf{Dates}\\
\midrule
Northeastern University & Assistant Professor & 2011--Present\\
\bottomrule
\end{tabularx}

\subsection*{Related Publications}

\begin{enumerate}
    \item Exploiting Execution Context for the Detection of Anomalous System Calls.
    Darren Mutz, William Robertson, Giovanni Vigna, and Richard Kemmerer.
    In Proceedings of the International Symposium on Recent Advances in Intrusion Detection (RAID).
    Gold Coast, Queensland AUS, September 2007.
    \item Polymorphic Worm Detection Using Structural Information of Executables.
    Christopher Kruegel, Engin Kirda, Darren Mutz, William Robertson, and Giovanni Vigna.
    In Proceedings of the International Symposium on Recent Advances in Intrusion Detection (RAID).
    Seattle, WA USA, September 2005.
    \item Automating Mimicry Attacks Using Static Binary Analysis.
    Christopher Kruegel, Engin Kirda, Darren Mutz, William Robertson, and Giovanni Vigna.
    In Proceedings of the USENIX Security Symposium.
    Baltimore, MD USA, July 2005.
    \item Detecting Kernel-Level Rootkits Through Binary Analysis.
    Christopher Kruegel, William Robertson, and Giovanni Vigna.
    In Proceedings of the Annual Computer Security Applications Conference (ACSAC).
    Tuscon, AZ USA, December 2004.
    \item Static Disassembly of Obfuscated Binaries.
    Christopher Kruegel, William Robertson, Fredrik Valeur, and Giovanni Vigna.
    In Proceedings of the USENIX Security Symposium.
    San Diego, CA USA, August 2004.
\end{enumerate}

\subsection*{Other Publications}

\begin{enumerate}
    \item PatchDroid: Scalable Third-Party Patches for Android Devices.
    Collin Mulliner, Jon Oberheide, William Robertson, and Engin Kirda.
    To appear in Proceedings of the Annual Computer Security Applications Conference (ACSAC).
    New Orleans, LA USA, December 2013.
    \item PrivExec: Private Execution as an Operating System Service.
    Kaan Onarlioglu, Collin Mulliner, William Robertson, and Engin Kirda.
    In Proceedings of the IEEE Symposium on Security and Privacy (Oakland).
    San Francisco, CA USA, May 2013.
    \item TRESOR-HUNT: Attacking CPU-Bound Encryption.
    Erik-Oliver Blass and William Robertson.
    In Proceedings of the Annual Computer Security Applications Conference (ACSAC).
    Orlando, FL USA, December 2012.
    \item Effective Anomaly Detection with Scarce Training Data.
    William Robertson, Federico Maggi, Christopher Kruegel, and Giovanni Vigna.
    In Proceedings of the Network and Distributed System Security Symposium (NDSS).
    San Diego, CA USA, February 2010.
    \item Are Your Votes Really Counted? Testing the Security of Real-world Voting Systems.
    Davide Balzarotti, Greg Banks, Marco Cova, Viktoria Felmetsger, William Robertson, Fredrik Valeur, Giovanni Vigna, and Richard Kemmerer.
    In Proceedings of the International Symposium on Software Testing and Analysis (ISSTA).
    Seattle, WA USA, July 2008.
\end{enumerate}

\subsection*{Synergistic Activities}

Dr.~William Robertson is an assistant professor of Computer Science at
Northeastern University in Boston, MA.  His research interests revolve around
the security of the web, mobile devices, and operating systems. He was
involved in both the California Top-to-Bottom-Review (TTBR) and the Ohio
EVEREST projects as a Red Team member.  In this capacity, he demonstrated that
electronic voting systems were potentially susceptible to large-scale attacks
that could exploit numerous vulnerabilities in the firmware and physical
security of the components of the voting system.  These findings led to
significant changes in public policy in both states with respect to electronic
voting, including restrictions or outright bans on the use of these systems.
He is also involved in Lastline, Inc., a startup focused on the detection of
advanced persistent threats and zero-day attacks that is based on the products
of his and his collaborators' research.

Dr.~Robertson also has extensive experience in organizing and participating in
Capture-the-Flag (CTF) exercises.  With Shellphish, a team composed of
UCSB-affiliated members, he won the 2005 edition of the DEFCON CTF
competition.  He was also instrumental in helping to organize the UCSB iCTF,
the largest distributed CTF competition, from its inception in 2003 to 2008.

Dr.~Robertson was the program co-chair of the 2013 USENIX Workshop on
Offensive Technologies (WOOT), co-located with USENIX Security.  He was the
chair of the 2012 Conference on the Detection of Intrusions and Malware \&
Vulnerability Assessment (DIMVA).  He has participated on the program
committees of a number of top-tier systems security venues, including IEEE
Security and Privacy, USENIX Security, and RAID.  He is also the author of
more than twenty peer-reviewed journal and conference papers in the area of
systems and network security.

\subsection*{Collaborators and Other Affiliations}

\noindent%
Davide Balzarotti (Institute Eurecom),
Leyla Bilge (Symantec Research Labs),
Erik-Oliver Blass (Northeastern University),
Ari Juels (RSA Labs),
Engin Kirda (Northeastern University),
Mohamed Kaafar (INRIA),
Christopher Kruegel (UC Santa Barbara),
Timothy Leek (MIT Lincoln Labs),
Todd Letham (RSA Labs),
Guevara Noubir (Northeastern University),
Jon Oberheide (Duo Security),
Alina Oprea (RSA Labs),
Theodoor Scholte (SAP),
Giovanni Vigna (UC Santa Barbara),
Ting-Fang Yen (RSA Labs).

\subsection*{Graduate Advisors and Postdoctoral Sponsors}

Giovanni Vigna and Richard Kemmerer (UC Santa Barbara), David Wagner (UC Berkeley).

\end{document}
